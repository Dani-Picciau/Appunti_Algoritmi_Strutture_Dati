\documentclass[../../main_document/main.tex]{subfiles}
\externaldocument{../../main_document/main}

\begin{document}
    \section{Hashing}
    L'\textbf{hashing} è una \textbf{tecnica alternativa} agli alberi binari di ricerca (capitolo \ref{par:Alberi binari di ricerca (BST)}) o agli array associativi, utilizzata per realizzare i dizionari.

    \vspace{8pt}
    \noindent
    A differenza dell'implementazione tramite alberi, nel quale si riusciva a mantenere la stessa complessità nei casi di \texttt{insert()}, \texttt{lookup()} e \texttt{remove()}, idealmente, la cosa migliore sarebbe \textbf{mantenere una complessità costante} per tutte le operazioni, avendo però una \textbf{complessità inferiore}. Questa implementazione ideale prende il nome di \textbf{tabelle di hash}.
    \begin{figure}[H]
        \centering
        \vspace{-10pt}  % Riduce lo spazio sopra
        \includegraphics[width=0.95\textwidth]{hashing/img1.png}
        \vspace{-5pt}  % Riduce lo spazio sopra 
    \end{figure}
    \noindent
    Per comprendere il funzionamento delle tabelle hash dobbiamo prendere in considerazione il concetto di \textbf{insieme universo \textit{U}}, ovvero un insieme di tutte le possibili chiavi, la cui grandezza dell'insieme varia arbitrariamente in base ai dati che si vogliono contenere.
    \begin{tcolorbox}[
        colback=yellow!20, 
        colframe=darkgray, 
        title=Idea di base
        ]
        Quello che si vuole fare è memorizzare tutti i dati dell'insieme \textbf{\textit{U}} in un vettore di dimensione ($m$) finita $T[0...m-1]$, ed avere un meccanismo per cui, data una chiave, si trovi rapidamente la posizione in cui è memorizzata.
    \end{tcolorbox}
    \noindent
    Le chiavi possono essere delle stringhe, degli oggetti o dei numeri, e il compito delle tabelle hash è quello di trasformarli in un indice all'interno delle tabelle hash. 
    Per fare ciò viengono utilizzate le \textbf{funzioni hash}.
    \begin{tcolorbox}[
        colback=yellow!20, 
        colframe=darkgray, 
        title=Cos'è una funzione hash?
        ]
        Quello che si vuole fare è memorizzare tutti i dati dell'insieme \textbf{\textit{U}} in un vettore di dimensione ($m$) finita $T[0...m-1]$, ed avere un meccanismo per cui, data una chiave, si trovi rapidamente la posizione in cui è memorizzata.
    \end{tcolorbox}
\end{document}